\documentclass[
    journal=jctcce,
    layout=twocolumn
]{achemso}

\setkeys{acs}{
	abbreviations = false,
	articletitle  = false,
	keywords      = false,
	maxauthors    = 10,
	super         = true
}

% Comment below before submitting:
\let\titlefont\undefined
\usepackage[fontsize=11pt]{scrextend}
\usepackage[hidelinks,colorlinks,citecolor=blue]{hyperref}
%\flushbottom
% Up to this point

\usepackage{amsmath}
\usepackage{amssymb}
\usepackage[T1]{fontenc}
%
\usepackage[table]{xcolor}
\definecolor{lightgray}{gray}{0.85}
%
\usepackage{array}
\newcolumntype{L}{>{$}l<{$}}
\newcolumntype{C}{>{$}c<{$}}
\newcolumntype{R}{>{$}r<{$}}
%
\newcommand{\mt}[1]{\boldsymbol{\mathbf{#1}}}   % matrix symbol
\newcommand{\vt}[1]{\boldsymbol{\mathbf{#1}}}   % vector symbol
\newcommand{\tr}[1]{#1^\text{t}}                % transposition
\newcommand{\diff}[2]{\frac{\partial #2}{\partial #1}} % derivative
%\newcommand{\diff}[2]{\nabla_{#1}{#2}} % derivative
%\newcommand{\diff}[2]{\partial_{#1}{#2}} % derivative
\newcommand{\avg}[1]{\overline{#1}}             % average

\newcommand{\dof}{i}   % index for each degree of freedom

%\listfiles

\author{Charlles R. A. Abreu}
\email{abreu@eq.ufrj.br}
\affiliation{Chemical Engineering Department, Escola de Quimica, Universidade Federal do Rio de Janeiro, Rio de Janeiro, RJ 21941-909, Brazil}
\alsoaffiliation{Department of Chemistry, New York University, New York, New York 10003, USA}

\author{Mark E. Tuckerman}
\email{marktuckerman@nyu.edu}
\affiliation{Department of Chemistry, New York University, New York, New York 10003, USA}
\alsoaffiliation{Courant Institute of Mathematical Sciences, New York University, New York, New York 10012, USA}
\alsoaffiliation{NYU-ECNU Center for Computational Chemistry at NYU Shanghai, Shanghai 200062, China}


\title{Free Energy Computation and Thermodynamic Property Reweighting From Multiple Time-Correlated Datasets}

\abbreviations{i.i.d., MC, MD, CLT, OBM, MSE, FEP, BAR, WHAM, MBAR, MICS}

\keywords{Free Energy Computation, Reweighting, Multistate, Uncertainty Estimation}

\begin{document}

%\begin{tocentry}
%Graphical Abstract
%\end{tocentry}

%\tableofcontents

\begin{abstract}
To be included.
\end{abstract}

\section{Introduction}
\label{sec:introduction}

\cite{Leimkuhler_2013}


\section{Results}


\section{Method}

\subsection{Equations of Motion}

Consider a system whose every degree of freedom $\dof$ has coordinate $q_\dof$, velocity $v_\dof$, and associated mass $m_\dof$.
The system is subject to a potential $U(\vt q)$ and, therefore, the force acting on $\dof$ is $f_\dof = -\diff{q_\dof}{U}$.
Also for every degree of freedom, we define auxiliary thermostat variables $v_{1,\dof}$ and $v_{2,\dof}$ with associated inertial parameters $ Q_1$ and $Q_2$, respectively.
The motion of this system is prescribed by a stochastic differential equation system which reads
\begin{subequations}
\begin{align}
& \dot{q}_\dof = v_\dof, \label{eq:q} \\
& \dot{v}_\dof = \frac{f_\dof}{m_\dof} - \lambda_\dof v_\dof, \label{eq:v} \\
& \dot{v}_{1,\dof} = - \lambda_\dof v_{1,\dof} - v_{2,\dof} v_{1,\dof}, \quad \mathrm{and}  \label{eq:v1} \\
& dv_{2,\dof} = \tfrac{Q_1 v_{1,\dof}^2 - kT}{Q_2}dt - \gamma v_{2,\dof} dt + \sqrt{\tfrac{2 \gamma kT}{Q_2}} dW_\dof, \label{eq:v2}
\end{align}
\end{subequations}
where $k$ is the Boltzmann constant,
$T$ is the temperature of the heat bath,
$\gamma$ is a friction constant,
%$\sigma = \sqrt{\frac{2 \gamma kT}{Q_2}}$,
$dW_\dof$ represents an infinitesimal increment of a Wiener process,
and $\lambda_\dof$ is a Lagrange multiplier aimed to enforce an isokinetic constraint to each degree of freedom, which is
\begin{equation}
m v_\dof^2 + \frac{1}{2} Q_1 v_{1,\dof}^2 = kT.
\end{equation}

Since it implies that $m v_\dof \dot{v}_\dof + \frac{1}{2} Q_1 v_{1,\dof}\dot{v}_{1,\dof} = 0$, substitution of Eqs.~\eqref{eq:v} and \eqref{eq:v1} into this constraint shows that
\begin{equation}
\lambda_\dof = \frac{f_\dof v_\dof - \frac{1}{2} Q_1 v_{2,\dof} v_{1,\dof}^2}{m_\dof v_\dof^2 + \frac{1}{2} Q_1 v_{1,\dof}^2}.
\end{equation}
        
In practice, a single value $Q_1 = Q_2 = kT\tau^2$, where $\tau$ is a relevant time scale for the system, is employed for all degrees of freedom.

\subsection{Solvation Free Energy}
\label{sec:solvation free energy}


\section{Conclusion}


\begin{acknowledgement}

C.R.A.A. acknowledges the financial support of Petrobras (project code CENPES 16113).

\end{acknowledgement}

\bibliography{sinr}

\end{document}